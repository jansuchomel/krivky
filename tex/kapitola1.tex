\chapter{Úvod}
Tato sbírka úloh vznikla jako maturitní práce na Smíchovské střední průmyslové škole ve školním roce 2013/2014
v rámci projektu Evropské unie OPPA ve spolupráci s Fakultou architektury Českého vysokého učení technického v Praze.	\\[10pt]
K matematice mám kladný vztah a proto jsem si vybral téma týkající se právě matematiky, konkrétně parametrický popis rovinných a prostorových křivek. Vzhledem k tomu, že k pochopení matematiky jsou důležité názorné příklady, rozhodl jsem se vytvořit sbírku řešených příkladů. Doplnil jsem též ukázky křivek z praxe. \\
Látka navazuje na středoškolskou matematiku, od čtenáře se předpokládá znalost analytické geometrie a znalost derivování. \\[10pt]
Text byl vytvořen v programu \LaTeX{}, obrázky rovinných křivek v programech \textit{Geogebra} a \textit{Sage Math}
a obrázky prostorových křivek byly vytvořeny v programovacím jazyku \textit{Python} za použití knihovny \textit{Matplotlib}. Další úpravy
obrázků byly provedeny ve vektorovém grafickém editoru \textit{Inkscape}. Všechen software použitý při tvorbě sbírky je volně
dostupný pod svobodnými licencemi. \\[10pt]